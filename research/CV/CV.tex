%------------------------------------
% Dario Taraborelli
% Typesetting your academic CV in LaTeX
%
% URL: http://nitens.org/taraborelli/cvtex
% DISCLAIMER: This template is provided for free and without any guarantee 
% that it will correctly compile on your system if you have a non-standard  
% configuration.
% Some rights reserved: http://creativecommons.org/licenses/by-sa/3.0/
%------------------------------------

%!TEX TS-program = xelatex
%!TEX encoding = UTF-8 Unicode

\documentclass[10pt, a4paper]{article}
\usepackage{fontspec} 
\usepackage[none]{hyphenat}
\usepackage{lastpage}


% DOCUMENT LAYOUT
\usepackage{geometry} 
\geometry{a4paper, textwidth=5.5in, textheight=8.5in, marginparsep=7pt, marginparwidth=.6in}
\setlength\parindent{0in}

% FONTS
\usepackage{xunicode}
\usepackage{xltxtra}
\defaultfontfeatures{Mapping=tex-text} % converts LaTeX specials (``quotes'' --- dashes etc.) to unicode
\setromanfont [Ligatures={Common}, BoldFont={Fontin Bold}, ItalicFont={Fontin Italic}]{Fontin}
\setsansfont [Ligatures={Common}, BoldFont={Fontin Sans Bold}, ItalicFont={Fontin Sans Italic}]{Fontin Sans}
\setmonofont[Scale=0.8]{Monaco} 
% ---- CUSTOM AMPERSAND
\newcommand{\amper}{{\fontspec[Scale=.95]{Fontin}\selectfont\itshape\&}}
% ---- MARGIN YEARS
\usepackage{marginnote}
\newcommand{\years}[1]{\marginnote{\scriptsize #1}}
\renewcommand*{\raggedleftmarginnote}{}
\setlength{\marginparsep}{7pt}
\reversemarginpar

% HEADINGS
\usepackage{sectsty} 
\usepackage[normalem]{ulem} 
\sectionfont{\rmfamily\mdseries\upshape\Large}
\subsectionfont{\rmfamily\bfseries\upshape\normalsize} 
\subsubsectionfont{\rmfamily\mdseries\upshape\normalsize} 

% PDF SETUP
% ---- FILL IN HERE THE DOC TITLE AND AUTHOR
\usepackage[xetex, bookmarks, colorlinks, breaklinks, pdftitle={Albert Einstein - vita},pdfauthor={My name}]{hyperref}  
\hypersetup{linkcolor=blue,citecolor=blue,filecolor=black,urlcolor=blue} 

% DOCUMENT
\begin{document}
{\LARGE Sagar Malhotra }\\[1cm]
 Fondazione Bruno Kessler\\
 Via Sommarive, 18\\
Trento, Trentino \texttt{38123}
Italy\\[.2cm]
Phone: \texttt{+39 320 841 2396}\\
email: \href{mailto:sagar.malhotra@unitn.it}{sagar.malhotra@unitn.it}\\[0.2cm]

\textsc{url}: \href{https://countinglogic.github.io}{countinglogic.github.io}\\[0.2cm] 

Born:  May 6, 1994\\
Nationality:  Indian

%%\hrule
\section*{Current position}
\years{11.2019-01.2023}\textsc{PhD Candidate}\\
Advisor: Luciano Serafini\\
Project: Analytical Approaches to Lifted Inference\footnote{Tentative title}\\
University of Trento, Italy\\
Fondazione Bruno Kessler, Italy

%\hrule
\section*{Research Experience and Education}
\noindent
\years{2018-2019}\textsc{Junior Research Fellow}\\
Advisors: Luciano Serafini, Radim Nedbal\\
Project: Variational Inference in Hybrid Domains\\
Data and Knowledge Management Unit\\
Fondazione Bruno Kessler, Italy\\

\years{2018-2015}\textsc{MSc} in Physics\\
Advisors: Roberto Iuppa (Unitn), Marco Cristoforetti (FBK)\\ 
Thesis: Deep Learning For Track Reconstruction in Next Generation HEP Experiments\\
Fondazione Bruno Kessler, Italy\\
University of Trento, Italy\\

\years{2015-2012}\textsc{BSc} in Physics with Honors\\
University of Delhi, India



\section*{Research Interests}
My current research interests revolve around knowledge representation and reasoning under uncertainity. I am especially interested in:\\

$\cdot$ Exact and Approximate Probabilistic Inference  \\
$\cdot$ Consistency of Probabilistic Inference \\
$\cdot$ Weighted Model Counting\\
$\cdot$ Structure Learning\\
$\cdot$ Exponential Random Graphs and their extension to logical domains\\
$\cdot$ Asymptotic Inference


\section*{Publications}

\noindent

\years{2022} Alessandro Daniele, Tommasso Campari, \textbf{Sagar Malhotra} and Luciano Serafini. \\ Deep Symbolic Networks: Discovering Symbols and Symbolic Knowledge from Perception \\ \emph{Under Review} \\ \\
\years{2022}\textbf{Sagar Malhotra} and Luciano Serafini. On Projectivity in Markov Logic Networks \\ \emph{Proceedings of Machine Learning and Knowledge Discovery in Databases. Research Track - European Conference, ECML PKDD 2022.} \\
\href{https://arxiv.org/pdf/2204.04009.pdf}{arXiv:2204.04009} \\ \\ 
\years{2022}\textbf{Sagar Malhotra} and Luciano Serafini. Weighted Model Counting in FO$^2$ with Cardinality Constraints and Counting Quantifiers: A Closed Form Formula \\ \emph{(\textbf{Oral} presentation) Proceedings of the $36^{th}$ AAAI Conference on Artificial Intelligence.}\\ \href{https://ojs.aaai.org/index.php/AAAI/article/view/20525}{AAAI 2022} \\

\years{2021}\textbf{Sagar Malhotra} and Luciano Serafini. A Combinatorial Approach to Weighted Model Counting in the Two Variable Fragment with Cardinality Constraints\\ \emph{ Proceedings of the $20^{th}$ International Conference of the Italian Association for Artificial Intelligence}\\ 
\href{https://link.springer.com/chapter/10.1007/978-3-031-08421-8_10}{AIxIA 2021}


\section*{Workshop Publications}
\noindent
\years{2022}\textbf{Sagar Malhotra} and Luciano Serafini. On Projectivity in Markov Logic Networks\\ \emph{$9^{th}$ International Workshop on Probabilistic Logic Programming, 2022}\\ 
\href{https://easychair.org/publications/preprint/2lTk}{PLP 2022:Preprint}\\ \\ 
\years{2021}\textbf{Sagar Malhotra} and Luciano Serafini. Weighted Model Counting in FO$^2$ with Cardinality Constraints and Counting Quantifiers: A Closed Form Formula\\ \emph{$10^{th}$ International Workshop on Statistical Relational AI, 2021}\\
\href{https://starai.cs.kuleuven.be/2021/}{StarAI 2021}



\section*{Talks and Tutorials}
\noindent

\years{2022} On Probabilistic Inference in Logical Domains\\
\emph{Invited Speaker at the Institute of Informatics, Ludwig Maximilian University of Munich, Germany}\\ 
\years{2022}A Tutorial on Probabilistic Inference in Logical Domains, University of Padova, Italy\\
\years{2022}Weighted First-Order Model Counting, DocInProgress, University of Trento, Italy \\
\years{2022}On Projectivity in Markov Logic Networks, DKM Group Seminar \\
\years{2022}Weighted First-Order Model Counting, AAAI 2022@FBK Workshop\\

%%\hrule
\section*{Programming Skills}
\noindent 
Fluent: Python, Pandas, \LaTeX\\
Familiar: Mathematica, R, Pytorch, HTML



\section*{Reviewing and PC Experience}
PC Member at AAAI 2022, Sub-Reviewer at KR 2021


\section*{Awards and Achievements}
%\years{2018}Bronze medal in TrackML particle tracking challenge on Kaggle.\\
\years{2017} Part of the winning team in Industrial Problem Solving using Physics (\href{https://www.unitn.it/archivio/events/ipsp2017.html}{IPSP 2017})\\
\years{2017}Awarded fully funded trip to Innovation Days-Innsbruck in StartUp Lab, Trento\\
\years{2016}Awarded full Scholarship for the Joint Masters in Theoretical Physics at University of Trento and SISSA- Trieste (Declined)\\
\years{2016}Awarded Opera Universitaria Scholarship for Masters in Physics at University of Trento\\
\years{2016}Amongst top 5$\%$ candidates in the Joint Entrance Screening Test- Physics 2016 among $\sim$ 5000 candidates\\
\years{2016}Amongst top 5 $\%$ candidates in IIT Joint Admission Test for Masters in Physics  2016 among $\sim$ 10000 candidates\\


\section*{References}
Luciano Serafini \\
Fondazione Bruno Kessler, Trento, Italy\\
Email: \href{mailto: serafini@fbk.eu}{serafini@fbk.eu} \\ \\

Dr. Felix Weitkämper \\
Institute of Informatics, LMU, Munich, Germany \\
Email: \href{mailto: felix.weitkaemper@lmu.de}{felix.weitkaemper@lmu.de} \\ \\ 

Dr. Mauro Dragoni \\
Fondazione Bruno Kessler, Trento, Italy\\
Email: \href{mailto: dragoni@fbk.eu}{dragoni@fbk.eu} 



% \section*{Service to the profession}

% ...

%\vspace{1cm}
\vfill{}
%\hrulefill

\begin{center}
{\scriptsize  Last updated: \today\- •\- 
% ---- FILL IN THE FULL URL TO YOUR CV HERE
\href{https://countinglogic.github.io/research/CV/CV.pdf}{For latest version click here. }}
\end{center}

\end{document}