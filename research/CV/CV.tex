%------------------------------------
% Dario Taraborelli
% Typesetting your academic CV in LaTeX
%
% URL: http://nitens.org/taraborelli/cvtex
% DISCLAIMER: This template is provided for free and without any guarantee 
% that it will correctly compile on your system if you have a non-standard  
% configuration.
% Some rights reserved: http://creativecommons.org/licenses/by-sa/3.0/
%------------------------------------

%!TEX TS-program = xelatex
%!TEX encoding = UTF-8 Unicode

\documentclass[10pt, a4paper]{article}
\usepackage{fontspec} 
\usepackage[none]{hyphenat}
\usepackage{lastpage}


% DOCUMENT LAYOUT
\usepackage{geometry} 
\geometry{a4paper, textwidth=5.5in, textheight=8.5in, marginparsep=7pt, marginparwidth=.6in}
\setlength\parindent{0in}

% FONTS
\usepackage{xunicode}
\usepackage{xltxtra}
\defaultfontfeatures{Mapping=tex-text} % converts LaTeX specials (``quotes'' --- dashes etc.) to unicode
\setromanfont [Ligatures={Common}, BoldFont={Fontin Bold}, ItalicFont={Fontin Italic}]{Fontin}
\setsansfont [Ligatures={Common}, BoldFont={Fontin Sans Bold}, ItalicFont={Fontin Sans Italic}]{Fontin Sans}
\setmonofont[Scale=0.8]{Monaco} 
% ---- CUSTOM AMPERSAND
\newcommand{\amper}{{\fontspec[Scale=.95]{Fontin}\selectfont\itshape\&}}
% ---- MARGIN YEARS
\usepackage{marginnote}
\newcommand{\years}[1]{\marginnote{\scriptsize #1}}
\renewcommand*{\raggedleftmarginnote}{}
\setlength{\marginparsep}{7pt}
\reversemarginpar

% HEADINGS
\usepackage{sectsty} 
\usepackage[normalem]{ulem} 
\sectionfont{\rmfamily\mdseries\upshape\Large}
\subsectionfont{\rmfamily\bfseries\upshape\normalsize} 
\subsubsectionfont{\rmfamily\mdseries\upshape\normalsize} 

% PDF SETUP
% ---- FILL IN HERE THE DOC TITLE AND AUTHOR
\usepackage[xetex, bookmarks, colorlinks, breaklinks, pdftitle={Albert Einstein - vita},pdfauthor={My name}]{hyperref}  
\hypersetup{linkcolor=blue,citecolor=blue,filecolor=black,urlcolor=blue} 

% DOCUMENT
\begin{document}
{\LARGE Sagar Malhotra }\\[0.1cm]
 Fondazione Bruno Kessler\\
 Via Sommarive, 18\\
Trento, Trentino \texttt{38123}
Italy\\
Phone: \texttt{+39 320 841 2396}\\
email: \href{mailto:smalhotra@fbk.eu}{smalhotra@fbk.eu}\\
\textsc{url}: \href{https://countinglogic.github.io}{countinglogic.github.io}\\
Born:  May 6, 1994\\
Nationality:  Indian
\section*{Personal Statement}
I am a graduate researcher in Artificial Intelligence, working on probabilistic inference and learning in relational domains (e.g., graphs, databases, logic, etc.). My work has focused on developing efficient, expressive, and statistically consistent probabilistic relational models. Previously, I pursued a masters in physics with an exciting mix of quantitative biology and machine learning. I am passionate about solving and formulating rigorous foundational problems with a multi-disciplinary flavor. 
I believe my background in logic, probability, physics and machine learning can allow me to greatly contribute to research in \\ experimental/behavioural economics.

%%\hrul
\section*{Current position}
\noindent
\years{11.2019-01.2023}\textsc{\textbf{PhD Candidate}}\\
Advisor: Prof. Luciano Serafini\\
Thesis: Towards Efficient and Consistent Probabilistic Inference in Relational Domains\footnote{Tentative title}\\
University of Trento, Italy\\
Fondazione Bruno Kessler, Italy\\
\underline{Achievements}:
\begin{itemize}
    \item Provided polynomial time closed form formulas for weighted model counting in the 2-variable fragment of first order logic and its extensions with cardinality constraints and counting quantifiers.
    \item Provided the first non-trivial fragment of Markov Logic Networks that admits consistent parameter estimation. Showed this fragment to be complete w.r.t the 2-variable Markov Logic.
    \item Provided an extended class of weight functions that admit efficient weighted model counting, expanding the expressivity of many probabilistic logic frameworks.
\end{itemize}


\section*{Research Experience and Education}
\noindent
\years{2018-2019}\textsc{\textbf{Junior Research Fellow}}\\
Advisors: Luciano Serafini, Radim Nedbal\\
Project: Variational Inference in Hybrid Domains\\
%Data and Knowledge Management Unit\\
Fondazione Bruno Kessler, Italy\\ 

\years{2018-2015}\textbf{MSc in Physics}\\
Advisors: Roberto Iuppa (Unitn), Marco Cristoforetti (FBK)\\ 
Thesis: Deep Learning For Track Reconstruction in Next Generation HEP Experiments\\
Fondazione Bruno Kessler, Italy\\
University of Trento, Italy\\

\years{2015-2012}\textsc{\textbf{BSc in Physics}}\\
University of Delhi, India



\section*{Publications}

\noindent

\years{2022} Alessandro Daniele, Tommasso Campari, \textbf{Sagar Malhotra} and Luciano Serafini. \\ Deep Symbolic Learning: Discovering Symbols and Rules from Perception \\ \emph{Under Review} \href{https://arxiv.org/abs/2208.11561}{arXiv:2208.11561}\\ \\
\years{2022}\textbf{Sagar Malhotra} and Luciano Serafini. On Projectivity in Markov Logic Networks \\ \emph{Proceedings of Machine Learning and Knowledge Discovery in Databases. Research Track - European Conference, ECML PKDD 2022} \\  \textbf{Largest European conference on machine learning with $\sim$1000 submissions and an \\ acceptance rate of $\sim$25\%} 
\href{https://arxiv.org/pdf/2204.04009.pdf}{ECML PKDD 2022}.
 \\ \\
\years{2022}\textbf{Sagar Malhotra} and Luciano Serafini. Weighted Model Counting in FO$^2$ with Cardinality Constraints and Counting Quantifiers: A Closed Form Formula \\ \emph{(\textbf{Oral} presentation) Proceedings of the $36^{th}$ AAAI Conference on Artificial Intelligence.}\\
\textbf{Flagship AI conference  with $\sim$10000 submissions and an  acceptance rate of  $\sim$ 10\% for oral presentations}
\href{https://ojs.aaai.org/index.php/AAAI/article/view/20525}{AAAI 2022} \\

\years{2021}\textbf{Sagar Malhotra} and Luciano Serafini. A Combinatorial Approach to Weighted Model Counting in the Two Variable Fragment with Cardinality Constraints\\ \emph{ Proceedings of the $20^{th}$ International Conference of the Italian Association for Artificial Intelligence}
\href{https://link.springer.com/chapter/10.1007/978-3-031-08421-8_10}{AIxIA 2021}


\section*{Workshop Publications}
\noindent
\years{2022}\textbf{Sagar Malhotra} and Luciano Serafini. On Projectivity in Markov Logic Networks\\ \emph{$9^{th}$ International Workshop on Probabilistic Logic Programming, 2022}\\ 
\href{https://easychair.org/publications/preprint/2lTk}{PLP 2022:Preprint}\\ \\ 
\years{2021}\textbf{Sagar Malhotra} and Luciano Serafini. Weighted Model Counting in FO$^2$ with Cardinality Constraints and Counting Quantifiers: A Closed Form Formula\\ \emph{$10^{th}$ International Workshop on Statistical Relational AI, 2021}\\
\href{https://starai.cs.kuleuven.be/2021/}{StarAI 2021}


% \section*{Research Interests}
% My research revolves around formal analysis of probability distributions over relational structures. I am especially interested in:

% $\cdot$ Random graph models and their extension to more complex relational domains\\
% $\cdot$ Exact and approximate probabilistic inference  \\
% $\cdot$ Combinatorics over relational structures\\ 
% $\cdot$ Consistency of probabilistic inference\\
% $\cdot$ Estimating asymptotic properties from sub-sampled relational structures\\



\section*{Talks and Tutorials}
\noindent

\years{2022} On Probabilistic Inference in Logical Domains\\
\emph{Invited Speaker at the Institute of Informatics, Ludwig Maximilian University of Munich, Germany}\\  \\ 
\years{2022}A Tutorial on Probabilistic Inference in Logical Domains\\ \emph{Guest Lecture at the Knowledge representation and Learning course, University of Padova, Italy}\\ \\
\years{2022}Weighted First-Order Model Counting \\
\emph{DocInProgress Colloquium, Department of Mathematics, University of Trento, Italy} \\ \\ 
\years{2022}Weighted First-Order Model Counting\\
\emph{AAAI 2022@FBK Workshop}

%%\hrule
\section*{Programming Skills}
\noindent 
Fluent: Python, Pandas, \LaTeX\\
Familiar: Mathematica, R, Pytorch, HTML



\section*{Reviewing and PC Experience}
PC Member at AAAI 2022, Reviewer at AISTATS 2022, Sub-Reviewer at KR 2021


\section*{Awards and Achievements}
%\years{2018}Bronze medal in TrackML particle tracking challenge on Kaggle.\\
\years{2017} Part of the winning team in Industrial Problem Solving using Physics (\href{https://www.unitn.it/archivio/events/ipsp2017.html}{IPSP 2017})\\
\years{2017}Awarded fully funded trip to Innovation Days-Innsbruck in StartUp Lab, Trento\\
\years{2016}Awarded full Scholarship for the Joint Masters in Theoretical Physics at University of Trento and SISSA- Trieste (Declined)\\
\years{2016}Awarded Opera Universitaria Scholarship for Masters in Physics at University of Trento\\
\years{2016}Amongst top 5$\%$ candidates in the Joint Entrance Screening Test- Physics 2016 among $\sim$ 5000 candidates\\
\years{2016}Amongst top 5 $\%$ candidates in IIT Joint Admission Test for Masters in Physics  2016 among $\sim$ 10000 candidates\\


\section*{References}
Luciano Serafini \\
Fondazione Bruno Kessler, Trento, Italy\\
Email: \href{mailto: serafini@fbk.eu}{serafini@fbk.eu} \\ \\

Dr. Felix Weitkämper \\
Institute of Informatics, LMU, Munich, Germany \\
Email: \href{mailto: felix.weitkaemper@lmu.de}{felix.weitkaemper@lmu.de} \\ \\ 

% Dr. Mauro Dragoni \\
% Fondazione Bruno Kessler, Trento, Italy\\
% Email: \href{mailto: dragoni@fbk.eu}{dragoni@fbk.eu} 



% \section*{Service to the profession}

% ...

%\vspace{1cm}
\vfill{}
%\hrulefill

\begin{center}
{\scriptsize  Last updated: \today\- •\- 
% ---- FILL IN THE FULL URL TO YOUR CV HERE
\href{https://countinglogic.github.io/research/CV/CV.pdf}{For latest version click here. }}
\end{center}

\end{document}